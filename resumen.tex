\chapter*{\large RESUMEN}
% \addcontentsline{toc}{chapter}{\large RESUMEN}

La integración de fuentes de datos es el problema de interconectar y acceder a fuentes heterogéneas de datos. En la medida en que las organizaciones han evolucionado este problema se ha convertido en un importante campo de investigación tanto para la academia como para la industria. El proyecto ``Las Tecnologías de la Información y la Comunicación apoyando los procesos educativos y la gestión del conocimiento en la educación superior'' (ELINF) requiere la integración de fuentes heterogéneas de datos para apoyar el proceso de control de autoridades en los sistemas informáticos que utiliza. La presente investigación tiene como objetivo desarrollar un método con componentes semánticos que permita la integración en una aplicación informática de datos relativos al control de autoridades, almacenados de forma heterogénea en las fuentes de datos utilizadas por el proyecto ELINF. En ella se describe el método desarrollado, a la vez que se valida por medio de un caso de estudio al comparar una aplicación desarrollada sin aplicar el método con una aplicación que instanció el método propuesto en cuanto a su capacidad para integrar fuentes heterogéneas de datos.

\textbf{Palabras clave:} Analíticas de datos, OBDA, OBDI, Ontologías, Semántica.

\textit{Data sources integration is about interconnecting and accessing heterogeneous data sources. With the evolution of the organizations, this topic has become an important research field both for academy and industry. The project ``Information and Communication Tecnologies supporting the educational process and the knowledge management in higher education'' (ELINF) needs to integrate heterogeneous data sources for supporting the authority control process in its software. This work aims to develop a method with semantic components to contribute to the integration of authority control related data required by the ELINF project stored in heterogeneous data sources  into a software. The current research describes the developed method. At the same time, it describes the study case used for validating it, through the comparison of a software application developed without the proposed method and another one instantiating the method regarding to its capacity to integrate heterogeneous data sources.}

\textit{\textbf{Keywords: }} \textit{Data analytics, OBDA, OBDI, Ontologies, Semantics}.