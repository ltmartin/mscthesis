\chapter{\large Método para la integración de datos basada en ontologías}


\pagestyle{fancy}
\lhead{}
\chead{}
\rhead{Capítulo 2: Método para la integración de datos basada en ontologías}
\lfoot{}
\cfoot{}
\rfoot{\thepage}
\renewcommand{\headrulewidth}{0.4pt}
%\renewcommand{\footrulewidth}{0.4pt}
 \vspace{-1cm}

\section{Paradigma utilizado en el desarrollo del método}
La investigación en la disciplina de sistemas de información (SI) se caracteriza por dos paradigmas: las ciencias del comportamiento y las ciencias del diseño \citep{Hevner:2004:DSI:2017212.2017217}. El primer paradigma persigue el desarrollo y verificación de teorías que expliquen o pronostiquen el comportamiento humano u organizacional. El paradigma de las ciencias del diseño tiene como fin la creación de innovaciones que definan ideas prácticas, capacidades tecnológicas y productos a través de los cuales puede lograrse el análisis, diseño, implementación, gestión y uso de sistemas de información de manera efectiva y eficiente \citep{Denning:1997:NSC:253671.253755}. Analizando la naturaleza del problema tratado en esta investigación y la relación entre su campo de acción (el mecanismo para integrar datos relativos al Control de Autoridades en la aplicación informática AUCTORITAS) y la disciplina de SI, el desarrollo de la solución se concibió y ejecutó bajo el paradigma de las ciencias del diseño.

Las ciencias del diseño crean y evalúan artefactos orientados a mejorar y entender el comportamiento de los sistemas de información, estos artefactos pueden ser constructos, modelos, métodos e instanciaciones \citep{March:1995:DNS:1700865.1700867}. Los constructos pertenecen al vocabulario conceptual de un dominio y son empleados por los modelos para representar una situación del mundo real en términos del diseño de un problema y su espacio de solución \citep{Simon:1996:SA:237774}. La búsqueda dentro de ese espacio de solución es guiada por métodos, los cuales pueden ser algoritmos matemáticos, descripciones textuales del proceso de búsqueda o combinaciones de ambas \citep{Hevner:2004:DSI:2017212.2017217}. Las instanciaciones demuestran la viabilidad de implementar los métodos y modelos, a la vez que facilitan la evaluación concreta del artefacto que representan.

La presente investigación fue desarrollada según el proceso definido por \cite{Peffers2006} para las investigaciones en ciencias del diseño. Este proceso propone seis actividades secuenciales: identificación del problema y motivación, objetivos de la solución, diseño y desarrollo, demostración, evaluación y comunicación. Los resultados de la primera y segunda actividad fueron expresados en la introducción de este documento. El diseño y desarrollo incluye determinar las funcionalidades deseadas del artefacto y su arquitectura para luego crear el artefacto en cuestión. Los elementos esenciales que debe poseer una aplicación informática para el OBDA/OBDI se sintetizan en el subepígrafe \ref{Acceso e Integración de Datos Basado en Ontologías}. Los artefactos
creados serán descritos en los epígrafes siguientes, mientras que la demostración de su eficacia en la solución del problema, así como la observación y medición de su desempeño relativos a las actividades cuatro y cinco del proceso serán expuestas en el Capítulo \ref{Capítulo 3}. Como evidencia de la comunicación a otros investigadores del problema abordado en la investigación y su relevancia, el artefacto creado, su utilidad y su efectividad, se encuentran artículos publicados en revistas y trabajos presentados en eventos por el autor de esta investigación.

