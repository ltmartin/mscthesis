\chapter{\large Características del sistema.}


\pagestyle{fancy}
\lhead{}
\chead{}
\rhead{Capítulo 2: Características del sistema}
\lfoot{}
\cfoot{}
\rfoot{\thepage}
\renewcommand{\headrulewidth}{0.4pt}
%\renewcommand{\footrulewidth}{0.4pt}
 \vspace{-1cm}


\lettrine[lines=2, lraise=0, nindent=0em, slope=-.5em]{{\bf O}}{}{\bf bjeto de estudio}


\begin{enumerate}
 \item Problema y situación problémica: 
	\begin{itemize}
	 \item Objetivos estratégicos de la organización y procesos de negocio que los soportan.
	 \item Flujo actual de los procesos involucrados en el campo de acción.
	 \item Análisis crítico de cómo se ejecutan actualmente esos procesos, las causas que originan la situación problémica y las consecuencias.
	\end{itemize}
 \item Objeto de automatización.
	\begin{itemize}
	 \item Descripción de los procesos que serán objeto de automatización. Debe incluir, si así lo requiere, las fórmulas a aplicar, los tipos de gráficos a representar, las estadísticas importantes que se deben obtener, entre otros aspectos. 
	 \item Descripción de los sistemas automatizados que existen en la empresa y que están vinculados con el campo de acción del proyecto.
	\end{itemize}
 \item Información que se maneja.
	\begin{itemize}
	 \item Documentos específicos que se procesen, detalles de la información que se manipule.
	\end{itemize}
 \item Propuesta de sistema.
	\begin{itemize}
	 \item Descripción general de la propuesta de sistema, como debe funcionar.
	 \item Análisis comparativo de otras soluciones existentes con la propuesta, llegando a conclusiones sobre los aspectos en los que su propuesta se diferencia de las otras y las supera.
	\end{itemize}
 \item Modelo de negocio. Ver anexo 1 \url{http://tesis.uci.cu/document/docguias/Guia1.doc}
 \item Especificación de los requisitos de software.
	\begin{itemize}
	 \item Dependencias y Relaciones con otros softwares. 
	 \item Requerimientos Funcionales (listado de las características básicas del sistema, enumeradas según la jerarquía que exista entre ellas)
	 \item Requerimientos no funcionales. Apariencia o interfaz externa.
		\subitem Usabilidad.
		\subitem Rendimiento.
		\subitem Soporte.
		\subitem Portabilidad. 
		\subitem Seguridad (Revisión bibliográfica del tema, especificar para el sistema la propuesta de: controles de seguridad y privacidad, seguridad física, controles administrativos y requerimientos funcionales que genera). En caso de que la empresa objeto de estudio cuente con políticas de seguridad bien establecidas y/o sistemas informáticos de seguridad, especificar cómo encaja en ellos el sistema propuesto.
		\subitem Políticos-culturales.
		\subitem Legales.
		\subitem Confiabilidad.
		\subitem Interfaz.
		\subitem Ayuda y documentación en línea.
      \end{itemize}
 \item Definición de los casos de uso. Ver anexo 3 \url{http://tesis.uci.cu/document/docguias/Guia1.doc}
\end{enumerate}