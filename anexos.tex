\renewcommand{\appendixname}{\large Anexo}
\appendix

\chapter{\large Glosario de términos}
\pagestyle{fancy}
\lhead{}
\chead{}
\rhead{Primer apéndice}
\lfoot{}
\cfoot{}
\rfoot{\thepage}
\renewcommand{\headrulewidth}{0.4pt}
\vspace{-1cm}

\textbf{Requisitos funcionales para datos de autoridad:} Modelo conceptual con el objetivo de proveer un marco de trabajo para el análisis de los datos de autoridad requeridos para soportar el Control de Autoridades y el intercambio internacional de datos de autoridad.

\textbf{Federación Internacional de Asociaciones de Bibliotecarios y Bibliotecas:} Organización mundial creada para proporcionar a los bibliotecarios un foro para intercambiar ideas, promoviendo la cooperación, la investigación y el desarrollo internacionales en todos los campos relacionados con la actividad bibliotecaria y la bibliotecología.



%%%%%%%%%%%%%%%%%%%%%%%%%%%%A PARTIR DE AQUÍ EL ANEXO 2

\chapter{\large Segundo apéndice}

\pagestyle{fancy}
\lhead{}
\chead{}
\rhead{Segundo apéndice}
\lfoot{}
\cfoot{}
\rfoot{\thepage}
\renewcommand{\headrulewidth}{0.4pt}
\vspace{-1cm}

\section{\large Definición de los casos de uso}

\subsection{\large Definición de actores}


  Representación de los casos de uso, indicando por qué lo son.
{%
  \newcommand{\mc}[3]{\multicolumn{#1}{#2}{#3}}
  \begin{longtable}{|l|l|}
  \hline
  \textbf{Actores} & \textbf{Justificación}\\
  \hline
  \mc{1}{|c|}{(1)} & \mc{1}{|c|}{(2)}\\
  \hline
  \end{longtable}
}%
  {\bf Leyenda:}
  \begin{enumerate}
   \item Nombre del actor.
   \item Breve descripción del actor.
  \end{enumerate}

  Listado de casos de uso.
  \begin{longtable}{|l|l|}
    \hline
    (1) & (2)\\
    \hline
    \textbf{Actor} & (3)\\
    \hline
    \textbf{Descripción} & (4)\\
    \hline
    \textbf{Referencia} & (5)\\
    \hline
  \end{longtable}

  {\bf Leyenda: }
\begin{enumerate}
 \item Código que se le de al caso de uso, nombre corto, por ejemplo CU-1.
 \item Nombre del caso de uso.
 \item Listado de actores que participa en el caso de uso.
 \item Descripción del caso de uso, esta descripción no tiene limites, por lo que se puede explicar bien todo 
lo que sucede, de manera general, sin llegar a detalles técnicos, es solo una descripción.
 \item Referencia a requerimientos funcionales o no funcionales, así como otros casos de uso, con los que esté 
relacionado el caso de uso.
\end{enumerate}

\subsection{Diagrama de casos de uso} 

Diagrama de casos de uso del sistema a automatizar. Se incluyen las relaciones de include, extend y 
generalización/especialización entre los casos de uso, y de generalización/especialización entre los actores, 
así como los casos de uso asociados con la seguridad. Estructurar los casos de uso en paquetes y poner el 
diagrama con las dependencias entre ellos.

{%
\newcommand{\mc}[3]{\multicolumn{#1}{#2}{#3}}
\definecolor{tcA}{rgb}{0.764706,0.764706,0.764706}
\begin{longtable}{|c|c|}
  \hline
% use packages: color,colortbl
  \mc{2}{|>{\columncolor{tcA}}c|}{\textbf{DIAGRAMA DE CASOS DE USO}}\\
  \hline
  \mc{2}{|c|}{(1)}\\
  \hline
\end{longtable}
}%

{\bf Leyenda:}
\begin{enumerate}
 \item Representación gráfica del diagrama.
\end{enumerate}

\subsection{Casos de uso por ciclo} 

Definir una tabla como la siguiente para cada uno de los ciclos de desarrollo.
\definecolor{tcA}{rgb}{0.764706,0.764706,0.764706}
  \begin{longtable}{|l|l|l|l|}
    \hline
    % use packages: color,colortbl
    \rowcolor{tcA}
    \textbf{Cód} & \textbf{Nombre de caso de uso} & \textbf{Paquete} & \textbf{Justificación de la selección}\\
    \hline
   &  &  & \\
  \hline
\end{longtable}

\subsection{Casos de uso expandidos}
{%
\newcommand{\mc}[3]{\multicolumn{#1}{#2}{#3}}
\definecolor{tcA}{rgb}{0.764706,0.764706,0.764706}
\begin{longtable}{|l|l|}
  \hline
  % use packages: color,colortbl
  \mc{2}{|>{\columncolor{tcA}}l|}{\textbf{Caso de uso}}\\
  \hline
  (1) & (2)\\
  \hline
  \textbf{Propósito} & (3)\\
  \hline
  \mc{2}{|l|}{\textbf{Actores}}\\
  \hline
  \mc{2}{|l|}{\textbf{Resumen: (4)}}\\
  \hline
  \textbf{Referencias} & (5)\\
  \hline
  \textbf{Acción del actor} & \textbf{Respuesta del sistema}\\
  \hline
  (6) & (7)\\
  \hline
  \mc{2}{|l|}{\textbf{Flujo alternativo (8)}}\\
  \hline
  \textbf{Acción del actor} & \textbf{Respuesta del sistema}\\
  \hline
  (9) & (10)\\
  \hline
  \mc{2}{|l|}{\textbf{Puntos de extensión}}\\
  \hline
  \mc{2}{|l|}{(11)}\\
  \hline
\end{longtable}
}%

{\bf Leyenda:}
\begin{enumerate}
 \item Código del caso de uso
 \item Nombre del caso de uso
 \item Objetivo del caso de uso.
 \item Descripción del caso de uso.
 \item Requerimientos funcionales o casos de uso con los que se relacione el caso de uso, ya sea porque es un requerimiento que le da origen, o por ser un caso de uso que se incluye o extiende.
 \item Listado de acciones que realiza el actor.
 \item Listado de respuestas del sistema ante las acciones del actor. {En esta parte se especifica el flujo básico o curso normal de las acciones, de existir condicionantes o posibles caminos se especifican en los Flujos alternativos.}
 \item Nombre o número del flujo alternativo
 \item Listado de acciones que realiza el actor.
 \item Listado de respuestas del sistema ante las acciones del actor.
 \item Listado de extensiones de los casos de uso. Especificar la línea del curso de acciones en que se encuentra esta extensión. Por ejemplo:
Línea 5 se autentica el usuario. Ver CU Autenticar usuario.
\end{enumerate}


 {\bf Nota:} Documentos que complementan el cuerpo del trabajo, pero que no son indispensables para 
entenderlo. Todos deben ser referenciados en el cuerpo del documento.


