\chapter{\large Análisis y diseño del sistema.}

\pagestyle{fancy}
\lhead{}
\chead{}
\rhead{Capítulo 3: Análisis y diseño del sistema}
\lfoot{}
\cfoot{}
\rfoot{\thepage}
\renewcommand{\headrulewidth}{0.4pt}
%\renewcommand{\footrulewidth}{0.4pt}
\vspace{-1cm}

\begin{itemize}
 \item \lettrine[lines=2, lraise=0, nindent=0em, slope=-.5em]{A}{}nálisis.
    \subitem Definición del modelo de análisis. Modelo de clases de análisis.
\begin{longtable}[c]{|l|}
\hline
\rowcolor{gray}
\multicolumn{1}{|>{\columncolor{Light}}c|}{DIAGRAMA DE CLASES DE ANÁLISIS}\\
\hline
\multicolumn{1}{|c|}{(1)}\\
\hline
\end{longtable}
    {\bf Leyenda}
    \begin{enumerate}
     \item Representación gráfica del diagrama.
    \end{enumerate}

\item Diagramas de interacción.

Por cada realización de casos de uso, diagramas de interacción (puede escogerse entre diagrama de secuencia 
y de colaboración) para flujo principal y flujos alternativos, y diagrama de clases (WEB, si es una 
aplicación web)

\begin{longtable}[c]{|l|}
\hline
\rowcolor{gray}
\multicolumn{1}{|>{\columncolor{Light}}c|}{DIAGRAMA DE INTERACCIÓN}\\
\hline
\multicolumn{1}{|c|}{(1)}\\
\hline
\end{longtable}
    {\bf Leyenda}
    \begin{enumerate}
     \item Representación gráfica del diagrama.
    \end{enumerate}


\begin{longtable}[c]{|l|}
\hline
\rowcolor{gray}
\multicolumn{1}{|>{\columncolor{Light}}c|}{DIAGRAMA DE CLASES}\\
\hline
\multicolumn{1}{|c|}{(1)}\\
\hline
\end{longtable}
{\bf Leyenda}
    \begin{enumerate}
     \item Representación gráfica del diagrama.
    \end{enumerate}


    \subitem Descripción de las clases.
    {%
  \newcommand{\mc}[3]{\multicolumn{#1}{#2}{#3}}
  \begin{longtable}{|l|l|}
      \hline
      \mc{2}{|l|}{\textbf{Nombre: (1)}}\\
      \hline
      \mc{2}{|l|}{\textbf{Tipo de clase (interfaz, controladora, entidad--)}}\\
      \hline
      \textbf{Atributo} & \textbf{Tipo}\\
      \hline
      (2) & (3)\\
      \hline
      \mc{2}{|l|}{\textbf{Para cada responsabilidad:}}\\
      \hline
      Nombre: & (4)\\
      \hline
      Descripción: & (5)\\
      \hline
    \end{longtable}
{\bf Leyenda}
\begin{enumerate}
 \item Nombre de la clase.
 \item Nombre de cada uno de los atributos.
 \item Tipo de dato de cada uno de los atributos.
 \item Nombre de la responsabilidad (operación) con los parámetros que requiera.
 \item Breve explicación de en qué consiste la responsabilidad.
\end{enumerate}

}%
  \subitem Diseño de la BD.
    \begin{itemize}
     \item Diagrama Entidad Relación de la BD.
    \end{itemize}
    \begin{longtable}[c]{|l|}
      \hline
      \rowcolor{gray}
      \multicolumn{1}{|>{\columncolor{Light}}c|}{DIAGRAMA ENTIDAD RELACIÓN}\\
      \hline
      \multicolumn{1}{|c|}{(1)}\\
      \hline
    \end{longtable}

    \begin{itemize}
     \item Descripción de las tablas.
    \end{itemize}
  {%
    \newcommand{\mc}[3]{\multicolumn{#1}{#2}{#3}}
    \begin{longtable}{|l|l|l|}
    \hline
    \mc{3}{|l|}{\textbf{Nombre: (1)}}\\
    \hline
    \mc{3}{|l|}{\textbf{Descripción: (2)}}\\
    \hline
    \textbf{Atributo} & \textbf{Tipo} & \textbf{Descripción}\\
    \hline
    (3) & (4) & (5)\\
    \hline
      \end{longtable}
}%
{\bf Leyenda}
\begin{enumerate}
 \item Nombre de la tabla.
 \item Descripción general de qué datos almacena la tabla.
 \item Nombre de cada uno de los atributos.
 \item Tipo de dato de cada uno de los atributos.
 \item Breve explicación qué es ese atributo.
\end{enumerate}

\item Definiciones de diseño que se apliquen. 
\item Tratamiento de errores.
\item Seguridad.
\item Interfaz.
\item Concepción de la ayuda.
\end{itemize}