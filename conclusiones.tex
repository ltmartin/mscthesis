\chapter*{\large Conclusiones generales}
\addcontentsline{toc}{chapter}{\large Conclusiones generales}

%% \addcontentsline{toc}{part}{Conclusiones}
%\pagestyle{fancy}
%\lhead{}
%\chead{}
%\rhead{\MakeUppercase{\textit{ Conclusiones}}}
%\lfoot{}
%\cfoot{}
%\rfoot{\thepage}
%\renewcommand{\headrulewidth}{0.4pt}
% \vspace{-1cm}

%\lettrine[lines=3]{L}

La realización de la presente investigación ratificó la necesidad de un método que conduzca el proceso de integración de datos almacenados en fuentes heterogéneas de datos y que aporte semántica al proceso. A su vez, se confirmó la Integración de Datos Basada en Ontologías como una de las vías más promisorias para la realización del proceso de integración de datos. Por otra parte, la utilización del paradigma de las ciencias del diseño resultó efectiva en el desarrollo del método propuesto. El método OntoIntegra permitió la integración de datos almacenados en fuentes estructuralmente heterogéneas, mientras que el caso de estudio utilizado para la validación del método propuesto permitió certificar su efectividad, por medio de la instanciación de la propuesta en una aplicación informática.
