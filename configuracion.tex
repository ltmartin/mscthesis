% Cambie el numero de la faculdad si es necesario.
\newcommand{\numeroFaculdad}{2}

% Escribir el titulo de la tesis que va a exponer.
\newcommand{\tituloTesis}{Método para la integración de datos basado en ontologías\\ aplicado al Control de Autoridades}

% Cambiar el tipo de trabajo si es necesario.
\newcommand{\nombreTipoTesis}{Trabajo final presentado en opción al título de\\ Máster en Informática Avanzada}

% Cambiar los nombres de los autores.
\newcommand{\autorUNO}{Ing. Leandro Tabares Martín}
\newcommand{\autorDOS}{Nombre Autor 2}

% Cambiar el nombre del tutor y co-tutor.
\newcommand{\tutor}{Dr.C. Yanio Hernández Heredia}
\newcommand{\coTutor}{Dr.C. Amed Abel Leiva Mederos}

% Cambiar el cargo del tutor y co-tutor
\newcommand{\cargoTutor}{Cargo del tutor}
\newcommand{\cargoCoTutor}{Cargo del co-tutor}

% Cambiar la fecha de confección de la tesis. Por defecto es el día de hoy
\newcommand{\fecha}{\large{julio de 2018}}

\newcommand{\espacios}{\vspace{0.5in}}
\newcommand{\logo}{\espacios\includegraphics{img/logo}\espacios}

\newcommand{\universidad}{\large{Universidad de las Ciencias Informáticas}}
\newcommand{\facultad}{\large{Facultad \numeroFaculdad}}

\newcommand{\titulo}{\large{\bf \tituloTesis}\espacios}

\newcommand{\tipoTesis}{\large{\nombreTipoTesis}\espacios}

\newcommand{\autores}
{
    \begin{center}
    \begin{tabular}{rl}
    \large{\bf Autor:}  & \large{\autorUNO} \\[0.7in]
    \large{\bf Tutor:}    & \large{\tutor}    \\
    \large{\bf Co-Tutor:} & \large{\coTutor}
    \end{tabular}
    \end{center}
    \vspace{1.0in}
}

\newcommand{\ciudadFecha}{\large{\bf Ciudad de la Habana, \fecha}}

\newcommand{\fillDia}{\makebox[0.3in]{\hrulefill}\space}
\newcommand{\fillMes}{\makebox[1in]{\hrulefill}\space}
\newcommand{\fillAnno}{\makebox[0.5in]{\hrulefill}}

\newcommand{\firma}{\makebox[2in]{\hrulefill}}

\newcommand{\firmaTesis}
{
    \begin{center}
    \begin{tabular}{cp{0.5in}c}
        \firma        \\
        \autorUNO     \\[1in]
        
    \end{tabular}
    \end{center}
}

\newcommand{\comentario}[2][]
{
	\todo[caption={#2}, size=\small, #1, inline,color={red!100!green!8},bordercolor=red,linecolor=red]
	{
	  \renewcommand{\baselinestretch}{1.2}\selectfont#2\par
	}
}
%Control de viudas y huérfanas(líneas de principio o fin de párrafo que se quedan solas al final o 
%al inicio de una página)
\clubpenalty=10000
\widowpenalty=10000

\definecolor{azul}{rgb}{0.53,0.81,0.98}


\newcommand\BackgroundPortada{
   \put(0,0){
     \parbox[b][\paperheight]{\paperwidth}{
       \vfill
       \centering
       \includegraphics[width=\paperwidth,height=\paperheight,%
                        keepaspectratio]{img/portada}%
       \vfill
     }}}

% \newcommand\BackgroundPortada{%
%  \put(0,0){%
%    \makebox(0,0)[bl]{%
%      \includegraphics[
%        width=\paperwidth,
%        height=\paperheight,
%        keepaspectratio
%      ]{img/portada}%
%    }%
% }%
% }
